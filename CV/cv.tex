%%%%%%%%%%%%%%%%%%%%%%%%%%%%%%%%%%%%%%%%%
% Medium Length Professional CV
% LaTeX Template
% Version 2.0 (8/5/13)
%
% This template has been downloaded from:
% http://www.LaTeXTemplates.com
%
% Original author:
% Trey Hunner (http://www.treyhunner.com/)
%
% Important note:
% This template requires the resume.cls file to be in the same directory as the
% .tex file. The resume.cls file provides the resume style used for structuring the
% document.
%
%%%%%%%%%%%%%%%%%%%%%%%%%%%%%%%%%%%%%%%%%

%----------------------------------------------------------------------------------------
%	PACKAGES AND OTHER DOCUMENT CONFIGURATIONS
%----------------------------------------------------------------------------------------

\documentclass{resume} % Use the custom resume.cls style

\usepackage[left=0.75in,top=0.6in,right=0.75in,bottom=0.6in]{geometry} % Document margins
%\usepackage{color,hyperref}
\usepackage[colorlinks = true,
            linkcolor = blue,
            urlcolor  = blue,
            citecolor = blue,
            anchorcolor = blue]{hyperref}
\usepackage{enumitem}
\usepackage{fancyhdr}
\usepackage{lastpage}

%\pagestyle{fancy}
%\fancyhf{}
\rfoot{Page \thepage \hspace{1pt} of \pageref{LastPage}}


\name{Zhiguang (Caleb) Huo} % Your name


\address{(352)-294-5929 \\ \url{zhuo@ufl.edu}} % Your phone number and email
\address{\url{https://caleb-huo.github.io}} % Your phone number and email
\address{(352)-294-5929 \\ \url{zhuo@ufl.edu}} % Your phone number and email
\address{2004 Mowry Road, 5th Floor CTRB, P.O. Box 117450, Gainesville, FL 32611-7450} % Your address

\begin{document}

%----------------------------------------------------------------------------------------
%	EDUCATION SECTION
%----------------------------------------------------------------------------------------
\footnotetext{Last modified: \today}

\begin{rSection}{Education}
\href{http://www.pitt.edu}{\textbf{University of Pittsburgh}}, \hfill{Pittsburgh, PA, US}
\begin{itemize}[noitemsep,topsep=0pt]
\item Ph.D. in
        \href{http://www.publichealth.pitt.edu/biostatistics}
             {Biostatistics},
              \hfill{\em April 2017}
	\begin{itemize}
        \item Dissertation:  \emph{Statistical integrative omics methods for disease subtype discovery}
        \item GPA: \emph{3.93/4.00}
        \item Advisors:
              \href{http://www.pitt.edu/~ctseng/}
                   {George C. Tseng, ScD} and
              \href{http://www.publichealth.pitt.edu/home/directory/yong-seok-park/}
                   {Yong Seok Park, PhD}
	\end{itemize}
	\item M.S. in
        \href{http://www.physicsandastronomy.pitt.edu}
             {Physics}, \hfill {\em Apr 2012}              
	\begin{itemize}
	\item {\emph{GPA: 3.86/4.00}}
	\end{itemize}
\end{itemize}

\href{http://www.hit.edu.cn} {\textbf{Harbin Institute of Technology}},
\hfill{Harbin, Heilongjiang, China}
\begin{itemize}[noitemsep,topsep=0pt]
\item B.S. in
        \href{http://physics.hit.edu.cn/}
             {Physics}, \hfill {\em June 2011} 
        \begin{itemize}
        \item GPA: \emph{90.43/100}
        \end{itemize}

\end{itemize}
\end{rSection}
%----------------------------------------------------------------------------------------
%	WORK EXPERIENCE SECTION
%----------------------------------------------------------------------------------------
\begin{rSection}{Research Interest}
My research interest lies in the intersection between statistical methodology and its applications to genomics and bioinformatics.	
I am particularly interested in genomic data integration, models and variable selection in high-dimensional data, graphical models, Bayesian methods, optimization and statistical computing. 
I have collaborated with biologists in the fields of cancer and psychiatry, analyzing a broad range of genomic data. 
These experiences motivate me to develop methodology and software that are practical, user-friendly and easy to use.
\end{rSection}

\begin{rSection}{Professional Experience}
\begin{itemize}[noitemsep,topsep=0pt]
\item Assistant Professor
        \hfill {\em July 2017 $\sim$ Now} 
        \begin{itemize}
        \item Department of Biostatistics, University of Florida
        \end{itemize}
\end{itemize}
\end{rSection}

%------------------------------------------------
\begin{rSection}{Publications}
\textbf{2017}
\begin{enumerate}[noitemsep,topsep=0pt]
\item  {\bf  Zhiguang Huo}, Shaowu Tang, Yongseok Park*, George Tseng*. (2017)
P-value evaluation, variability index and biomarker categorization for adaptively weighted Fisher's meta-analysis method in omics applications. (Submitted)	

\item Marianne Seney,  {\bf Zhiguang Huo}, Leon French, Rachel Puralewski, Joyce Zhang, David A Lewis, George Tseng, Etienne Sibille. (2017) 
Distinct molecular signatures of depression in men and women. (Submitted)

\item {\bf Zhiguang Huo}, Chi Song, George C. Tseng. (2017)
Bayesian latent hierarchical model for transcriptomic meta-analysis to detect biomarkers with clustered meta-patterns of differential expression signals. Submitted to \emph{Annals of Applied Statistics} (under revision).


\item {\bf Zhiguang Huo}, George C. Tseng. (2017)
    Integrative Sparse $K$-means with overlapping group lasso in genomic applications for disease subtype discovery.
    \emph{The Annals of Applied Statistics}, 11(2), 1011-1039.


    \item  Dominique Arion, {\bf Zhiguang Huo}, John F. Enwright, John P. Corradi, George Tseng and David A. Lewis. (2017)
Transcriptome alterations in prefrontal pyramidal neurons distinguish schizophrenia from bipolar and major depressive disorders.
\emph{Biological Psychiatry}, 

\item SungHwan Kim, Dongwan Kang, {\bf Zhiguang Huo}, Yongseok
Park, George C. Tseng. (2017)
Meta-analytic principal component analysis in integrative omics application.
(minor revision).

\item Abraham Apfel,  {\bf Zhiguang Huo}, George Tseng and Stewart J. Anderson (2017)
Achieving Accurate and Stable Feature Selection via Sparse K-means. (submitted)


\item Enwright, John, {\bf Zhiguang Huo}, Dominique Arion, John Corradi, Aiqing He, George Tseng, and David Lewis. (2017) 
Transcriptome alterations of prefrontal cortical parvalbumin neurons in schizophrenia. (accepted).


\end{enumerate}

\textbf{2016}
\begin{enumerate}[noitemsep,topsep=0pt,resume]

 \item  {\bf Zhiguang Huo}, Ying Ding, Silvia Liu, Steffi Oesterreich, and George Tseng. Meta-Analytic Framework for Sparse $K$-Means to Identify Disease Subtypes in Multiple Transcriptomic Studies. \emph{Journal of the American Statistical Association},  111, no. 513 (2016): 27-42.

\item Zhu, Li, Ying Ding, Cho-Yi Chen, Lin Wang, {\bf Zhiguang Huo}, SungHwan Kim, Christos Sotiriou, Steffi Oesterreich, and George C. Tseng. "MetaDCN: meta-analysis framework for differential co-expression network detection with an application in breast cancer." \emph{Bioinformatics} (2016): btw788.




\end{enumerate}


\textbf{2015 and before}
\begin{enumerate}[noitemsep,topsep=0pt,resume]
    \item Silvia Liu, Wei-Hsiang Tsai, Ying Ding, Rui Chen, Zhou Fang, {\bf Zhiguang Huo}, SungHwan Kim, Tianzhou Ma, Ting-Yu Chang, Nolan Michael Priedigkeit, Adrian V. Lee, Jianhua Luo, Hsei-Wei Wang, I-Fang Chung, George C. Tseng. (2015).
Comprehensive evaluation of fusion transcript detection algorithms and a meta-caller to combine top performing methods in paired-end RNA-seq data.
\emph{Nucleic Acids Research}, 10.1093/nar/gkv1234.

    \item Tiffany A. Katz, Serena G. Liao, Vincent J. Palmieri, Robert K. Dearth, Thushangi Pathiraja, {\bf Zhiguang Huo}, Patricia Shaw, Sarah Small, Nancy E. Davidson, David G. Peters, George C. Tseng, Steffi Oesterreich, Adrian V. Lee. (2015) Targeted DNA Methylation Screen in the Mouse Mammary Genome Reveals a Parity-Induced Hypermethylation of IGF1R That Persists Long after Parturition. \emph{Cancer Prevention Research} 8, no. 10 (2015): 1000-1009.

    \item Yan P. Yu, Silvia Liu, {\bf Zhiguang Huo}, Amantha Martin, Joel B. Nelson, George C. Tseng and Jian-Hua Luo. (2015) Genomic copy number variations in the genomes of leukocytes predict prostate cancer clinical outcomes. \emph{PloS one}, 10(8):e0135982.

\item SungHwan Kim,  {\bf Zhiguang Huo}, YongSeok Park and George Tseng.  (2015) MetaOmics: transcriptomic meta-analysis methods for biomarker detection, pathway analysis and other exploratory purposes. Book chapter in Integrating omics data: statistical and computational methods. Edited by George C. Tseng, Debashis Ghosh, Xianghong Jasmine Zhou. \emph{Cambridge University Press}. Page 39-67.

    \item Xingbin Wang, Dongwan Kang, Kui Shen, Chi Song, Shuya Lu, Lunching Chang, Serena G. Liao, {\bf Zhiguang Huo}, Naftali Kaminski, Etienne Sibille, Yan Lin, Jia Li and George C. Tseng. (2012) A Suite of R Packages for Quality Control, Differentially Expressed Gene and Enriched Pathway Detection in Microarray Meta-analysis. \emph{Bioinformatics}, 28:2534-2536.

\end{enumerate}

\textbf{Under preparation}
\begin{itemize}
\item  YongSeok Park, {\bf Zhiguang Huo}, Shaowu Tang and George Tseng. (2017)
Asymptotic properties of adaptive weighted Fisher's method.
\item Li Zhu, {\bf Zhiguang Huo}, Tianzhou Ma, George Tseng. (2017)
Bayesian indicator variable selection model with multi-layer overlapping groups.
\item Tianzhou Ma$^*$, {\bf Zhiguang Huo}$^*$, Anche Kuo$^*$, Xiangrui Zeng, Li Zhu, Ark Fang, Lin Wang, Chien-Wei Lin, Tanbin Rahman, Shuchang Liu, YongSeok Park, Sunghwan Kim, Jia Li, Lun-Ching Chang, Chi Song, George Tseng.  (2017)
MetaOmics - a Comprehensive Software Suite with Interactive Visualization for Transcriptomic Meta-Analysis.
($*$: co-first author).

\item  George C. Tseng, {\bf Zhiguang Huo} and Tianzhou Ma. (2017)
Foundations for High-Throughput Omics Data Analysis: Methods, Theories and Applications. \emph{Chapman \& Hall/CRC}. 

\end{itemize}

\end{rSection}

\begin{rSection}{Award}
\textbf{Student Awards}
\begin{itemize}[noitemsep,topsep=0pt]
\item  Delta Omega Membership \hfill April 2017
\item American Statistics Association (ASA) Pittsburgh chapter  \hfill March 2016
\begin{itemize}[noitemsep,topsep=0pt]
\item Student of the year
\end{itemize}
\item Department of Physics, Harbin Institute of Technology \hfill May 2009
\begin{itemize}[noitemsep,topsep=0pt]
\item National Scholarship of P.R. China. \\(Awarded to the top 2 students in my Bachelors degree.)
\end{itemize}
\end{itemize}


\textbf{Travel Awards} 
\begin{itemize}[noitemsep,topsep=0pt]
\item SAMSI Research Triangle Park, NC.	
\begin{itemize}[noitemsep,topsep=0pt]
\item {Interface of Statistics and Optimization} \hfill Feb 2017
\item {Optimization Summer School} \hfill Aug 2016
\item {Epigenetics Workshop} \hfill Mar 2015
\item {Beyond Bioinformatics Workshop} \hfill June 2014
\end{itemize}

\end{itemize}

\end{rSection}

%----------------------------------------------------------------------------------------
%	TECHNICAL STRENGTHS SECTION
%----------------------------------------------------------------------------------------

\begin{rSection}{Teaching Experience (University of Pittsburgh)}
\textbf{Main Lecturer (teaching fellow)}
\begin{itemize}[noitemsep,topsep=0pt]
\item BIOST2094 - Advanced R Computing -- (with Tianzhou Ma) \hfill Spring 2017
\item BIOST2025 - Special Studies in Bayesian Data Analysis  \hfill Fall 2016
\begin{itemize}[noitemsep,topsep=0pt]
\item{(with George Tseng, Tianzhou Ma and Li Zhu)}
\end{itemize}
\end{itemize}


\textbf{Guest Lecturer}
\begin{itemize}[noitemsep,topsep=0pt]
\item BIOST2055 - Introductory high-throughput genomic data analysis I: \\data mining and applications \hfill Mar 2016
\begin{itemize}[noitemsep,topsep=0pt]
\item{Differential and isoform analysis of RNA-seq data}
\end{itemize}
\item BIOST2078 - Introductory high-throughput genomic data analysis II: \\theories and algorithms  \hfill Dec 2015
\begin{itemize}[noitemsep,topsep=0pt]
\item{Reproducible research and parallel computing in R}
\end{itemize}
\item BIOST2078 - Introductory high-throughput genomic data analysis II: \\theories and algorithms  \hfill Dec 2014
\begin{itemize}[noitemsep,topsep=0pt]
\item{Reproducible research}
\end{itemize}
\end{itemize}

\textbf{Teaching Assistant}
\begin{itemize}[noitemsep,topsep=0pt]
\item BIOST 2078 - Introductory high-throughput genomic data analysis II: \\theories and algorithms \hfill {Fall 2014}
\item PHYS 0212 - Introduction to Laboratory Physics \hfill {Spring 2012}
\item PHYS 0212 - Introduction to Laboratory Physics \hfill {Fall 2011}
\end{itemize}

\end{rSection}

\begin{rSection}{Presentations}
\textbf{Poster and Oral Presentation}

\begin{itemize}[noitemsep,topsep=0pt]
\item  Poster, Dean's Day's competition, GSPH, University of Pittsburgh \hfill {April 2017}
\begin{itemize}[noitemsep,topsep=0pt]
\item{Circadian rhythms of gene expression in the human prefrontal cortex \\reveal distinct pattern between schizophrenia and control subjects}
\end{itemize}

\item  Invited talk, University of Florida, Gainesville, FL \hfill {Feb 2017}
\begin{itemize}[noitemsep,topsep=0pt]
\item{Meta-analytic and integrative framework for sparse <i>K</i>-means to identify disease subtypes.}
\end{itemize}

\item Poster, SAMSI optimization summer school,  Research Triangle Park, NC  \hfill {Aug 2016}
\begin{itemize}[noitemsep,topsep=0pt]
\item{Integrative Sparse $K$-means for disease subtype discovery using \\multi-level omics data.}
\end{itemize}

\item Poster, Pittsburgh ASA banquet, Pittsburgh, PA \hfill {Mar 2016}
\begin{itemize}[noitemsep,topsep=0pt]
\item{Integrative Sparse $K$-means for disease subtype discovery using \\multi-level omics data.}
\end{itemize}

\item Oral Presentation, JSM, Seattle, WA \hfill {Aug 2015}
\begin{itemize}[noitemsep,topsep=0pt]
\item{Meta-analytic framework for sparse $K$-means to identify disease\\ subtypes in multiple transcriptomic studies.}
\end{itemize}

\item Poster, Pittsburgh ASA banquet, Pittsburgh, PA \hfill {Apr 2015}
\begin{itemize}[noitemsep,topsep=0pt]
\item{Meta-analytic framework for sparse $K$-means to identify disease\\ subtypes in multiple transcriptomic studies.}
\end{itemize}

\item Oral Presentation, ENAR Conference, Miami, FL \hfill {Mar 2015}
\begin{itemize}[noitemsep,topsep=0pt]
\item{Meta-analytic framework for sparse $K$-means to identify disease\\ subtypes in multiple transcriptomic studies.}
\end{itemize}

\item Poster, Dean's Day's competition, GSPH, University of Pittsburgh \hfill {Mar 2015}
\begin{itemize}[noitemsep,topsep=0pt]
\item{Discover and Characterize Invasive Lobular Breast Carcinoma Subtypes.}
\end{itemize}

\item Oral Presentation, ENAR Conference, Baltimore, MA \hfill {Mar 2014}
\begin{itemize}[noitemsep,topsep=0pt]
\item{Meta-analytic framework for sparse $K$-means to identify disease\\ subtypes in multiple transcriptomic studies.}
\end{itemize}

\item Poster, Dean's Day's competition, GSPH, University of Pittsburgh \hfill {Mar 2014}
\begin{itemize}[noitemsep,topsep=0pt]
\item{Meta-analytic framework for sparse $K$-means to identify disease \\ subtypes in multiple transcriptomic studies.}
\end{itemize}
\end{itemize}

\textbf{Seminar Talk}
\begin{itemize}[noitemsep,topsep=0pt]
\item Department of Biostatistics, University of Pittsburgh  \hfill Nov 2015
\begin{itemize}[noitemsep,topsep=0pt]
\item{How to use Latex to make slides}
\end{itemize}
\end{itemize}


\end{rSection}

%\begin{rSection}{References}
%\textbf{George C. Tseng}
%\begin{itemize}[noitemsep,topsep=0pt]
%\item[] Professor \hfill {Phone: 412-624-5318}\\
%Department of Biostatistics (primary appointment) \hfill{E-mail: ctseng@pitt.edu }\\
%Department of Human Genetics \\
%Department of Computational \& Systems Biology 
%\\
%University of Pittsburgh
%\end{itemize}

%\textbf{Yong Seok Park}
%\begin{itemize}[noitemsep,topsep=0pt]
%\item[] Assistant Professor \hfill {Phone: 412-624-3028}\\
%Department of Biostatistics \hfill{E-mail: yongpark@pitt.edu}\\
%University of Pittsburgh
%\end{itemize}

%\textbf{David A. Lewis, MD}
%\begin{itemize}[noitemsep,topsep=0pt]
%\item[] Distinguished Professor of Psychiatry and Neuroscience, \\
%Thomas Detre Professor of Academic Psychiatry, Chair  
%\hfill {Phone: 412-246-6010}\\
%Department of Psychiatry \hfill{E-mail: lewisda@upmc.edu}\\
%University of Pittsburgh
%\end{itemize}

%\end{rSection}
\begin{rSection}{Reviewer servise}
\begin{itemize}[noitemsep,topsep=0pt]
\item PLOS ONE
\item Scientific Reports
\end{itemize}
\end{rSection}


\begin{rSection}{Membership}
\begin{itemize}[noitemsep,topsep=0pt]
\item Member of International Chinese Statistical Association
        \hfill {\em Mar 2015 $\sim$ Now} 
\item Member of Eastern North American Region International Biometric Society
        \hfill {\em Oct 2013 $\sim$ Now} 
\item Member of American Statistical Association
        \hfill {\em Nov 2013 $\sim$ Now} 
\end{itemize}
\end{rSection}


\begin{rSection}{Hobbies}
Running, skiing, other endurance activities.

\textbf{Running Record}
\begin{itemize}[noitemsep,topsep=0pt]
    \item Bank of America Chicago Marathon, Chicago, IL. \hfill {10/09/2016}
    \item First National Bank Pittsburgh Triathlon (Sprint), Pittsburgh, PA \hfill {08/14/2016}
    \item Dick's Sporting Goods Pittsburgh Marathon, Pittsburgh, PA. \hfill {05/01/2016} 
    \item First National Bank Pittsburgh Triathlon (Olympic standard), Pittsburgh, PA \hfill {08/09/2015}
    \item {Dick's Sporting Goods Pittsburgh Marathon}, {Pittsburgh, PA} \hfill  {05/03/2015}
    \item {Dick's Sporting Goods Pittsburgh Marathon}, {Pittsburgh, PA} \hfill  {{05/04/2014}}
    \item {Dick's Sporting Goods Pittsburgh Marathon}, {Pittsburgh, PA} \hfill  {05/05/2013}
\end{itemize}

\end{rSection}


%----------------------------------------------------------------------------------------
%	EXAMPLE SECTION
%----------------------------------------------------------------------------------------

%\begin{rSection}{Section Name}

%Section content\ldots

%\end{rSection}

%----------------------------------------------------------------------------------------

\end{document}
